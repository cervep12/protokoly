% ----------------------------------------------------------------------
%  Pracovní úkoly - opište přímo ze zadání
% ----------------------------------------------------------------------
\section{Pracovní úkoly}

\begin{enumerate}


\item Určete relativní permitivitu (dielektrickou konstantu) materiálu pomocí proložení změřené závislosti.

\end{enumerate}

% ----------------------------------------------------------------------
%  Použité pomůcky
% ----------------------------------------------------------------------


\section{Použité přístroje a pomůcky}
	Hliníková fólie, neznámý materiál, multimetr s možností měření kapacity, vodiče s krokosvorkami, posuvné měřidlo

% ----------------------------------------------------------------------
%  Teoretický úvod - vlastními slovy stručne popište fyzikální podstatu měření a uveďte základní vztahy použité ve vypracování
% ----------------------------------------------------------------------


% ----------------------------------------------------------------------
%  Postup měření - vlastními slovy popište postup měření tak, aby bylo vaše měření reprodukovatelné 
% ----------------------------------------------------------------------
	\section{Měření}
	Určili jsme tloušťku neznámého materiálu pomocí posuvného měřidla. Vytvořili jsme 2 čtverce z hliníkové fólie. Tyto 2 čtverce jsme dali z každé strany na neznámý materiál. Ke čtvercům jsme připojili krokosvorky s multimetrem. Změřili jsme kapacitu vzniklého kondenzátoru. Velikosti folií jsme poté měnili. 
	
	\subsection{Výsledky měření: }Výsledná tloušťka neznámého předmětu:  $d=(0.5\pm 0.01)mm$. Velikosti stran fólií v cm:  5, 10, 20, 30, 40, 50 a jim odpovídající výsledné kapacity v nanofaradech: 0.315,  0.912, 3.290,  11.141, 17.022, 25.118. Chyba měření kapacity je 20\% ze změřené hodnoty.

					
			
% ----------------------------------------------------------------------
%  Naměřené hodnoty a samotné vypracování úkolu
% ----------------------------------------------------------------------				
		\section{Vypracování}
		\subsection{Matematika}
		Rovnice \eqref{eq:1} slouží pro výpočet kapacity kondenzátoru. Pomocí rovnice \eqref{eq:2} vypočteme obsah čtverců z hliníkové fólie.
		V rovnici \eqref{eq:3} provedeme substituci veličin C a d. V rovnici \eqref{eq:4} určíme chybu měření veličiny y. Dostaneme lineární rovnici \eqref{eq:5}, kde členy na levé straně nemají žádnou chybu měření. Určíme relativní permitivitu dle \eqref{eq:6}.
		
		\begin{equation}\label{eq:1}
			C = \epsilon \frac{S}{d}
		\end{equation}
		
		\begin{equation}\label{eq:2}
		S_i = a_i^2
		\end{equation}
		
		\begin{equation}\label{eq:3}
			y_i = C_i d
		\end{equation}
			
		
	\begin{equation}\label{eq:4}
	\sigma_{y_i} = y_i \sqrt{\left( \frac{\sigma_C}{C}\right) ^2 + \left( \frac{\sigma_d}{d}\right) ^2}
	\end{equation}	
\begin{equation}\label{eq:5}
	y_i = \epsilon S_i
\end{equation}
\begin{equation}\label{eq:6}
\epsilon_r = \frac{\epsilon}{\epsilon_0}
\end{equation}

		
		\subsection{Naměřené hodnoty}
Měření jsme zpracovali do následující Tabulky \ref{tab:table1}.
Směrnice přímky na Obrázku \ref{fig:pic} udává permitivitu prostředí mezi deskami.
		\begin{center}
		
	\begin{table}[h]
		\begin{center}
			\begin{tabular}{|l|S|S|S|S|}
	\hline
				\# & $C[nF]$ & $S [cm^{2}]$& $y[nF \cdot m]$& $\sigma_{y_i}[nF \cdot m]$ \\
\hline 
				1 & 0.315 & 25& 0.016& 0.003\\
				2 & 0.912 & 100& 0.046& 0.009\\
				3 & 3.290 & 400& 0.165& 0.030\\
4 &  11.141& 900& 0.557& 0.100\\
 5&  17.022& 1600& 0.850& 0.200\\
 6&  25.118 & 2500& 1.256& 0.300\\
 \hline
			\end{tabular}
		\caption{Tabulka kapacit, obsahů a veličiny y s chybou}
		\label{tab:table1}
		\end{center}
	\end{table}		
	\end{center}




\begin{figure}[!htp]
	\centering
	\includegraphics[keepaspectratio,width=15cm,height=15cm]{img/permitivita_ZFM.eps}
	\caption{Závislost hodnot y na obsahu desek}
	\label{fig:pic}
\end{figure}
\newpage
		\begin{center}
	
	\begin{table}[h]
		\begin{center}
			\begin{tabular}{|l|l|}
				\hline
				statistiky & hodnoty \\
				\hline 
				fitting method & leastsq \\
				data points   &6\\
				variables & 2 \\
				chi-square&  2.705\\
				reduced chi-square&  0.676\\
				Akaike info crit&  -0.780 \\
 A& $4.792 \cdot 10^{-11} \pm 4.012 \cdot 10^{-12}$\\
B & $2.995 \cdot 10^{-14} \pm 2.872 \cdot 10^{-14}$\\
				\hline
			\end{tabular}
			\caption{Statistiky grafu}
			\label{tab:table2}
		\end{center}
	\end{table}		
\end{center}
Relativní permitivitu dostaneme z rovnice \eqref{eq:6}. Vydělíme hodnotu A z Tabulky \ref{tab:table2} permitivitou vakua.  




% ----------------------------------------------------------------------
%  Diskuse - obsahuje komentář k jednotlivým výsledkům, porovnání s očekáváním/tabulkovými hodnotami, zdroje především systematických chyb měření, návrh na zlepšení výsledků,...
% ----------------------------------------------------------------------			
	

	\section{Diskuse}
Na malé množství naměřených hodnot dostáváme poměrně slušný fit.
I přesto jsou zde některé nepřesnosti. Hodnota relativní permitivity nejvíce odpovídá relativní permitivitě skla, která má hodnotu 4-8.   Dali by se zlepšit použitím profesionálnějších laboratorních pomůcek. Každopádně zpracování dat, které jsem snad nepokazil, bych viděl jako hlavní těžiště tohoto cvičení. 

			

		

% ----------------------------------------------------------------------
%  Závěr - stručně a jasně shrnout splněné cíle měření, úkoly a výsledky měření
% ----------------------------------------------------------------------
			
\section{Závěr}
		Výsledná relativní permitivita materiálu mezi deskami kondenzátoru je: \newline \begin{center}
			$\epsilon_r =5.4 \pm 0.5$
		\end{center}
		
	



